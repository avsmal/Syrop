\documentclass[unicode]{beamer}

% utf-8 damnit!
\usepackage[utf8]{inputenc}
\usepackage[russian]{babel}

%\usetheme{Pittsburgh}
%\usecolortheme{beaver}

\usetheme{Madrid} % My favorite!

%\usetheme{Boadilla} % Pretty neat, soft color.
%\usetheme{default}
%\usetheme{Warsaw}
%\usetheme{Bergen} % This template has nagivation on the left
%\usetheme{Frankfurt} % Similar to the default 
%with an extra region at the top.
%\usecolortheme{seahorse} % Simple and clean template
%\usetheme{Darmstadt} % not so good
% Uncomment the following line if you want %
% page numbers and using Warsaw theme%
% \setbeamertemplate{footline}[page number]
%\setbeamercovered{transparent}
\setbeamercovered{invisible}
% To remove the navigation symbols from 
% the bottom of slides%
\setbeamertemplate{navigation symbols}{} 
%

\usepackage{graphicx}
%\usepackage{bm}         % For typesetting bold math (not \mathbold)
%\logo{\includegraphics[height=0.6cm]{yourlogo.eps}}
%
\title[Syrop]{Syrop: Автоматическая конфигурация настроек прокси для Linux}
\author{Алексей Великий, Михаил Кринкин, Ксения Крашенинникова, Семен Мартынов, Александр Смаль}
\institute[APTU]
{
Санкт-Петербугский Академический Университет
 \\
\medskip
{\emph{avsmal@gmail.com}}
}
\date{\today}
% \today will show current date. 
% Alternatively, you can specify a date.
%
\begin{document}
%
\begin{frame}
\titlepage
\end{frame}
%
\begin{frame}
\frametitle{Идея}
\begin{itemize}
\item 
Многие приложения используют подключение к Internet \\
(\emph{svn}, \emph{ssh}, \emph{bash}, etc.)
\item
Разные системы - разные интерфейсы настройки
\item
Хорошо бы иметь возможность определять настройки в одном
месте так, чтобы они автоматически применялись для
всех программ
\end{itemize}
\end{frame}

\begin{frame}
\frametitle{Постановка задачи}
Необходимо написать приложение, которое
\begin{itemize}
\item предоставляет единый интерфейс для  конфигурации разных систем
\item позволяет хранить настройки для разных сетей \\
и реагирует на смену сети
\item обладает расширяемостью: \\
можно добавить поддержку новых приложений
\item позволяет отменять примененные настройки

\end{itemize}
\end{frame}

\begin{frame}
\frametitle{Структура проекта}
\begin{itemize}
\item \emph{core} :  обеспечение интеграцию плагинов
\item \emph{plugins} :  расширения, выполняющие настройки подключения к сети для разных систем
\item \emph{network} : перехватывет сообщения D-BUS о состоянии сети
\item \emph{utils} : взаимодействие системы с файловой системой, 
парсер настроек сетей и локальных настроек системы
\item \emph{gui} : поддержка консольного и графического режима работы
\end{itemize}
\end{frame}
%
%example for ini format

\begin{frame}
\frametitle{Syrop}
\end{frame}

\begin{frame}
\frametitle{syropd}
\end{frame}

\begin{frame}
\frametitle{QT и GTK GUI}
\end{frame}

\begin{frame}

\frametitle{Plug-Ins}

\begin{itemize}

\item default - применяется, если для протокола не заданы специальные настройки

\item svn
\begin{itemize}
\item updates '~/.subversion/servers'
\item specifies HTTP settings
\end{itemize}

\item ssh
\begin{itemize}
\item ssh - updates '~/.ssh/config'
\item specifies HTTP settings
\end{itemize}

\item bash
\begin{itemize}
\item svn - updates '~/.bashrc'
\item specifies HTTP, FTP settings
\end{itemize}

\item gconf
\begin{itemize}
\item runs native gconftool configurator tool
\item specifies HTTP, FTP, SOCKS protocols settings
\end{itemize}

\end{itemize}

\end{frame}

\begin{frame}
\frametitle {Результаты}
\begin{itemize}
\item tar-ball package available at https://github.com/avsmal/Syrop
\item система сборки CMake
\item GUI: QT и GTK
\item реализованы плагины для svn, ssh, bash, gconf
\item расширяемость
\end{itemize}
\end{frame}

\begin{frame}
\frametitle {Развитие проектов}
\begin{itemize}
\item поддержка новых плагинов
\item расширение интерфейса
\end{itemize}
\end{frame}


%\end{Huge}
 
% End of slides
\end{document} 